\documentclass[a4paper,11pt,oneside,openany]{jsarticle}
    \usepackage[dvipdfmx]{graphicx}
    \usepackage{amsmath,amssymb}
    \usepackage{bm}
    \usepackage{graphicx}
    \usepackage{ascmac}

    \thispagestyle{empty}
    %
    \begin{document}
    \begin{center}

      \vspace*{35mm}
      \huge レポート3 \par
      \vspace{90mm}
      \Large 提出期限:2019/7/19\\
       提出日: \today \\
      \vspace{15mm}
      \Large
       1029293806\hspace{5mm}大山 偉永\par

      \vspace{10mm}
    \end{center}
    \clearpage
    \addtocounter{page}{-1}

    \newpage

\section{はじめに}
  このレポートでは計算機科学実験及演習3におけるインタプリタ作成実験の型推論実装部分の演習課題に対する説明を与える。セクションは各課題ごとに分かれておりそのセクション内部ではその課題に対する設計方針と実装の詳細/工夫した点の二つの項目に分かれている。

\section{Exercise4.2.1}
\subsection{各プログラムの設計方針}
  資料のコードをコピーし...の部分を自分で埋める。ty\_primのMultのパターンなどはPlusのパターンを参考にすれば実装できる。
  それ以外のExpの部分も他のすでに埋めてある部分を参考にして同じような共同を示すように埋めていく。
\subsection{実装の詳細/工夫した点}
    ty\_expのIfExp (exp1, exp2, exp3)のパターンの場合はまずexp1についてty\_expを再帰的に適用してその型を得て、その型がTyBoolかどうかで分岐する。
    もしTyBoolでないときはこれはifの文の制約を満たさないのでエラーを返す。exp1の型がTyBoolの時はif\ A\ then\ B\ else\ Cにおける
    BとCの部分の方は一致する必要があるのでそこの型が等しくない時はエラーを返す。等しい時は上の例の式におけるBの型を返す。\\
    次にLetExp (id, exp1, exp2) の時はまずexp1の型を今までと同様にty\_primを再帰的に適用して解析し、idでこのexp1の型を束縛してtyenvを拡張する。
    その上でexp2をty\_primによって評価する。\\\\



\section{Exercise4.3.1}
\subsection{各プログラムの設計方針}
    図4.1中の pp\_ty,与えられ た型中の型変数の集合を返す関数で、
    型は val\ freevar\_ty\ :\ ty\ -$>$\ tyvar\ MySet.t とする関数freevar\_ty を完成させる。freevar\_ty は型'a MySet.t は mySet.mli で定義されている'a を要素とする集合を表す型であるのでそれをもとに実装する。
\subsection{実装の詳細/工夫した点}
    pp\_tyはstring\_of\_tyをprint\_stringしたものなのでpp\_tyの実装はlet\ pp\_ty\ ty\  =\ print\_string\  (string\_of\_ty\ ty)とするだけでよい。
    次に与えられた型中の型変数(int)のリストを返す関数freevar\_tyについてはその引数がどのty型であるかによってパターンマッチを行う。
    引数がTyFun(a,b)の時はaとbについてそれぞれfreevar\_tyを適用して返されたリストを結合する。
    引数がTyVar\ aの時はそのままaをリストに追加して返せばいい。\\\\



\section{Exercise4.3.2}
\subsection{各プログラムの設計方針}
  型代入に関するtype\ subst\ =\ (tyvar * ty)\ list型、型代入(のリスト)をひとつの型に適用するval\ subst\_type\ :\ subst\ $->$\ ty\ ->\ ty型の関数を typing.ml 中に実装する。

\subsection{実装の詳細/工夫した点}
    型代入をtyに適用する関数を作る。tyの型によってパターンマッチングを行う。
    tyがTyFun(a,b)の時はこのaとbに再帰的にsubst\_typeを適用しその結果返されたリストをunionを用いて結合する。\\
    TyVar\ thetaのときはまず型代入の1つ目(型代入をhd::tlとしたときにhdの部分)の型変数alphaがこのthataと一致するときはこのTyVar\ thetaを型代入の一つ目の型変数alphaに対応する型で置き換えてそのTyVar\ (thetaをalphaに対応する型で置き換えた型)に再び型代入の残りのリストをsubst\_typeを用いて再帰的にこの関数を適用する。このthetaが型代入の一つ目の型変数alphaと異なる場合はその型に残りの型代入tlを再帰的に適用する。\\
    パターンマッチにおける型がTyFunでもTyVarでもない時は型代入を適用する部分はないのでそのまま与えられた型を返すように実装する。\\\\


\section{Exercise4.3.3}
\subsection{各プログラムの設計方針}
  資料内の単一化に関する部分を読み実装する。そのアルゴリズムについては詳細がすでに書かれているので基本的にはその方針に沿って実装していけばいい。単一化の関数の型はval\ unify\ :\ (ty * ty)\ list\ -$>$\ substである。

\subsection{実装の詳細/工夫した点}
  まず単一化を適用したい型の等式集合でパターンマッチを行いその型の等式集合が空リストの場合はそのまま空リストを返す。つぎにその型の等式集合が(ty1,ty2)::tlのようにかける時、つまり型の等式集合の一つ目の要素が(ty1,ty2)で残りの要素の集合がtlの時は以下の手順を実行する。\\
  まずty1とty2が既に等しい時は何もする必要がないので残りの型の等式集合の部分tlに再び再帰的にこの単一化の関数unifyを適用する。そうでないときはこの(ty1,ty2)でパターンマッチを行う。
  \begin{itemize}
    \item(ty1,ty2)が(TyFun(ty1,ty2),TyFun(ty3,ty4))の時はこのファンクションの引数の部分と式部分の型がそれぞれ一致する必要があるので(ty1,ty3)と(ty2,ty4)という等式集合を残りの等式集合tlに追加して再びこれにunifyを適用する。
    \item(ty1,ty2)が((TyVar\ a),ty)の時はtyに含まれる型変数にこのaが含まれているときはエラーを返し、含まれていないときは残りの型の等式集合tlにこの型代代入(a,ty)を適用してその等式集合に再びこの関数を適用する。ここで、型代入(のリスト)を一つの型に適用する関数subst\_typeは既に定義したが今回は一つの型代入を型の等式集合に適用したいので新しくsubst\_unifyという関数を定義する。これは今書いたように一つの型代入を型の等式集合に適用する関数でありsubst\_typeを使うように工夫することで簡単に定義できる。このように型代入を残りの等式集合tlに適用したものに再びunifyを適用した結果返ってくる型代入(のリスト)に今の型代入(a,ty)を追加する。
    \item(ty1,ty2)が(ty,TyVar a)の時は上の場合と逆のことを行えばいい。
    \item(ty1,ty2)が上記のいずれでもないときはエラーを返す。
  \end{itemize}
  以上が単一化のアルゴリズムの実装方式である。
\\\\

\section{Exercise4.3.4}
単一化アルゴリズムにおいて,α !∈ FTV(τ) という条件の必要性を考える。たとえば型の等式集合の要素として(α,α-$>$α)のようなものがあった際これはα∈FTV(τ)であり、\\α=α-$>$α= (α-$>$α)-$>$(α-$>$α)=$\cdot\cdot\cdot$のようにこの型の等式集合を解こうとすると無限ループしてしまう。したがってこのようにτに型変数αは含まれないという制約を付ける必要がある。
\\\\

\section{Exercise4.3.5}
\subsection{各プログラムの設計方針}
  まずは各型付け規則の型推論手続きを与える。
  \begin{itemize}
    \item 変数xに対して型付け規則T-VARの型推論の手続きを行う。
    \begin{enumerate}
      \item Γ,$x を入力として型推論を行いSと,τ$ を得る。
      \item $Sとτ$を出力として返す。
    \end{enumerate}
  \end{itemize}
  \begin{itemize}
    \item 整数nに対して型付け規則T-INTの型推論の手続きを行う。
    \begin{enumerate}
      \item Γ,$n$ を入力として型推論を行い空の型代入と,TyInt を得る。
      \item $\bigl[\ \bigr]$とTyIntを出力として返す。
    \end{enumerate}
  \end{itemize}
  \begin{itemize}
    \item 真偽値bに対して型付け規則T-BOOLの型推論の手続きを行う。
    \begin{enumerate}
      \item Γ,$b を入力として型推論を行いS_1と,τ$を得る。
      \item 型代入$S_1$ を α = τ という形の方程式の集まりとみなして,$S_1∪\{(τ,TyBool)\} を単一化し,型代入S_2$ を得る
      \item $S_2$とTyBoolを出力として返す。
    \end{enumerate}
  \end{itemize}
  \begin{itemize}
    \item 式$e_1*e_2$に対して型付け規則T-MULTの型推論の手続きを行う。
    \begin{enumerate}
      \item Γ,$e_1 を入力として型推論を行いS_1と,τ_1$を得る。
      \item Γ,$e_2 を入力として型推論を行いS_2と,τ_2$を得る。
      \item 型代入$S_1,S_2$ を α = τ という形の方程式の集まりとみなして,$S_1∪S_2∪\{(τ_1,TyInt),(τ_2,TyInt)\} を単一化し,型代入S_3$ を得る
      \item $S_3$とTyIntを出力として返す。
    \end{enumerate}
  \end{itemize}
  \begin{itemize}
    \item 式$e_1 <e_2$に対して型付け規則T-LTの型推論の手続きを行う。
    \begin{enumerate}
      \item Γ,$e_1 を入力として型推論を行いS_1と,τ_1$を得る。
      \item Γ,$e_2 を入力として型推論を行いS_2と,τ_2$を得る。
      \item 型代入$S_1,S_2$ を α = τ という形の方程式の集まりとみなして,$S_1∪S_2∪\{(τ_1,TyInt),(τ_2,TyInt)\} を単一化し,型代入S_3$ を得る
      \item $S_3$とTyBoolを出力として返す。
    \end{enumerate}
  \end{itemize}
  \begin{itemize}
    \item if文$if\ e_1\ then\ e_2\ else\ e_3$ に対して 型付け規則T-IFの型推論の手続きを行う。
    \begin{enumerate}
      \item Γ,$e_1 を入力として型推論を行いS_1と,τ_1$を得る。
      \item Γ,$e_2 を入力として型推論を行いS_2と,τ_2$を得る。
      \item Γ,$e_3 を入力として型推論を行いS_3と,τ_3$を得る。
      \item 型代入$S_1,S_2,S_3$ を α = τ という形の方程式の集まりとみなして,$S_1∪S_2∪S_3∪\{(τ_1,TyBool),(τ_2,τ_3)\} を単一化し,型代入S_3$ を得る
      \item $S_3とτ_2$を出力として返す。
    \end{enumerate}
  \end{itemize}
  \begin{itemize}
    \item let文$let\ x\ =\ e_1\ in\ e_2$ に対して型付け規則T-LETの型推論の手続きを行う。
    \begin{enumerate}
      \item Γ,$e_1 を入力として型推論を行いS_1と,τ_1$を得る。
      \item Γ,x:$τ_1,e_2 を入力として型推論を行いS_2と,τ_2$を得る。
      \item 型代入$S_1,S_2$ を α = τ という形の方程式の集まりとみなして,$S_1∪S_2 を単一化し,型代入S_3$ を得る。
      \item $S_3とτ_2$を出力として返す。
    \end{enumerate}
  \end{itemize}
  \begin{itemize}
    \item function式$fun\ x\ ->\ e$ に対して型付け規則T-FUNの型推論の手続きを行う。
    \begin{enumerate}
      \item xの型を$τ_1$とする。
      \item Γ,x:$τ_1,eを入力として型推論を行いSと,τ_2$を得る。
      \item $SとTyFun(τ_1,τ_2)$を出力として返す。
    \end{enumerate}
  \end{itemize}
  \begin{itemize}
    \item 関数の適用式$e_1\ e_2$ に対して型付け規則T-APPの型推論の手続きを行う。
    \begin{enumerate}
      \item Γ,$e_1を入力として型推論を行いS_1と,τ_1$を得る。
      \item Γ,$e_2を入力として型推論を行いS_2と,τ_2$を得る。
      \item $e_1の型をTyFun(τ_3,τ_4)とする。$
      \item 型代入$S_1,S_2$ を α = τ という形の方程式の集まりとみなして,$S_1∪S_2∪\{(τ_1,TyFun(τ_3,τ_4)),(τ_2,τ_3)\}  を単一化し,型代入S_3$ を得る。
      \item $S_3とτ_3$を出力として返す。
    \end{enumerate}
  \end{itemize}
  以上の型推論の手続きを実際に実装していく。実装についてはT-PLUS,T-FUN,T-VARはすでに実装してあるので残りの部分を例に倣って埋めていく方針で行う。
\subsection{実装の詳細/工夫した点}
  \begin{itemize}
    \item T-MULT,T-LT,はT-PLUSの場合とほぼ同じでty\_prim関数でその制約と型を返す。その制約の内容は乗算、比較ともにその二項の型はTyIntであるという型代入が必要で返す型は乗算の場合はTyInt、比較の場合はTyBoolを返す。
    \item T-IFの場合はty\_exp関数の引数がIfExpの時の箇所で実装する。上で記述した型推論の手続きと同様にif文if\ exp1\ then\ exp2\ else\ exp3におけるexp1,exp2,exp3の部分をまずty\_expで再帰的に評価しその結果返ってきた型ty1,ty2,ty3について上と同様の制約を型代入に追加しunifyで単一化し、その型代入と型ty2を返す。
    \item T-LETの場合もty\_exp関数の引数がLetExpの時の箇所で実装する。上で記述した型推論の手続きと同様にlet文let\ id\ =\ exp1\ in\ exp2におけるexp1部分をまずty\_expで再帰的に評価しその結果返ってきた型ty1で型環境tyenvを拡張する。また同時に返ってきた型代入s1とその拡張した型環境でexp2を評価しその結果返ってきた型代入s2を結合し単一化を行いexp2を評価した型ty2にこの型代入を適用してその型代入と型を返す。
    \item T-APPの場合もty\_exp関数の引数がAppExpの時の箇所で実装する。関数適用式exp1\ exp2においてまずexp1,exp2の型をty\_expで評価し(s1,ty1)と(s2,ty2)を得る。ここでこの ty1がTyFun型であるという制約を与えるためにfresh\_tyvarを用いて新たな型domty1.domty2を作りdomty1とty2が等しいという型代入とty1とTyFun(domty1,domty2)が等しいという型代入をs1とs2を結合したものに追加して単一化しその型代入と、domoty2にその型代入を適用した型subst\_type\ s4\ domty2を返す。\\\\
  \end{itemize}

\section{Exercise4.3.6}
\subsection{各プログラムの設計方針}
  資料に与えられているlet\ rec式の型付け基礎に基づいてその型推論部分を実装する。今まで同様にこの型付け規則を忠実に表現する実装を行えば再現できる。
\subsection{実装の詳細/工夫した点}
  let\ rec式let\ rec\ id\ =\ fun exp1 -$>$\ exp2においてまずidの型に対応する関数の型で型環境を拡張する必要があるがこの時点でこの関数の型の内容は分からないので自分で新たにfresh\_tyvarを用いて型変数を二つ用意しその型domty1,domty2を一度そのidに対する型をTyFun(domty1,domty2)として拡張しその環境をnewenv1とする。またparaに関しても同様に今作った型domty2で型環境newenv1を拡張しこの型環境をnewenv2とする。このnewenv1でexp2を、newenv2でexp1をty\_expで評価し(s1, ty1) と(s2, ty2)を得る。
  このs1とs2を結合しこれにdomty2とty1が等しいという制約を加え単一化しこの型代入と、この型代入をty2に適用した型を返す。
\\\\

\section{Exercise4.4.1}
\subsection{各プログラムの設計方針}
  多相的let 式・宣言ともに扱える型推論アル ゴリズムの実装を完成させる。資料内で実装済みの部分はそこを写経する。資料中に作るべき関数などは書いてあるのでまずはその関数を作りそこから対応部分を変えていく方針で実装した。

\subsection{実装の詳細/工夫した点}
  まず関数freevar\_tyの拡張で、型スキームσ を受け取り、σ に自由に出現する(ただし型環境には出現しない)型変数 の集合を計算する関数であるfreevar\_tyscを実装する。まず型スキームの情報の中にその型を束縛した変数の情報がありその自由変数の集合をlstとする。次にその型スキームに含まれるの型の情報をty1としてfreevar\_tyを用いてty1に含まれる自由変数を求めその集合をtyvarlstとする。よって求める集合はこのtyvarlstの集合からlstを引いたものである。ここでlstはlist型で求める集合はMySet型なのでMySet.from\_listを用いて一度lstの型を変更しそこでさ集合を求める。\par
  この関数を用いてfreevar\_tyenvを定義する。この関数は与えられた型環境の中にある、型スキーム内の束縛されていない自由変数の集合を返すものである。型環境をまずlist型に変換するためにEnvironment.mlでこのenv型からlist型に変換するto\_listを定義しこの引数のtyenvをリストに変換する。次にそのリストについてパターンマッチを行う。空リストの場合はMySet型の空リストを返す。型環境が(\_,tysc)::tl と表せるときはまずそのtyscについて先ほど定義したfreevar\_tyscを用いてその自由変数を求め残りのtlの部分についても同様にこの関数freevar\_tyenvを再帰的に適用する。ここで残りのtlについて再帰的にこの関数を適用する際にこの関数の引数はenv型であるためこのlist型のtlは引数にとれないためここで再びこのtlをenv型に変換する関数.from\_listを定義した。以上で与えられた定義すべき関数の実装は終わりである。\par
  残りはty\_expの環境を拡張する部分を型スキームで拡張するように変更する。 LetExp (id, exp1, exp2)の部分は今までexp1の型でidを束縛して型環境を拡張していたのを、closureを用いてexp1の型τ と型環境 tyenvとexp1の評価で得た型代入s1 から、条件「α1,...,αn は τ に自由に出現する型変数で s1Γ には自由に出現しない」を満たす型スキーム ∀α1.··· .∀αn.τ を求めその型スキームでidを束縛し環境を拡張するように変更する。FunExp (id, exp)については関数の自分で作った引数の型domty1でidを束縛して型環境を拡張していたが、これを型スキームTyScheme([],domty)で束縛するようにして拡張するように変更した。以上でこの問の実装は終わりでLetRecExpの多相型に対応する型推論の実装は次の演習問題で述べる。
\\\\

\section{Exercise4.4.2}
  \subsection{各プログラムの設計方針}
    再帰関数を多層型で型推論できるように拡張する。問題文で型付け規則は与えられており、これにそって型推論できるように実装していく。
  \subsection{実装の詳細/工夫した点}
    まず上の問題と同様に環境の拡張を型で拡張するのではなくその型スキームで行うようにする。以下具体的に述べる。LetRecExp(id, para, exp1, exp2)においてparaの型をfresh\_varを用いてdomty1とし同様にしてexp1の型をdomty2とする。このとき多層型での型推論を実装する前はまずidをTyFun(domty1,domty2)で束縛して環境を拡張していたのを、TyScheme($\bigl[\bigr]$,(TyFun(domty1,domty2))で束縛して拡張してnewenv1を定義する。paraについて束縛の拡張も同様でparaをdomty1で拡張していたのをTyScheme($\bigl[\bigr]$,domty1)で束縛してnewenv1を拡張するようにする。この環境をnewenv2とする。\par
    次にこのnewenv2環境下でexp1を評価し(s1,ty1)を得る。またここでdomty2とty1は等しいという制約があるためここで一度これと制約s1を結合し単一化し型代入s5を得る。次にexp2の評価のためにidを多相的なTyFun( domty1,ty1)で束縛して拡張した環境で評価する必要があるのでこのTyFun( domty1,ty1)にs5を適用しこの型、型環境tyenv、型代入s5を引数にclosureを用いて型スキームを得る。この型スキームでidを束縛しtyenvを拡張しえたnewenv3でexp2を評価する。こうして得た型代入s2を今までの型の制約に追加し単一化し最後にty2にこの型代入を適用して得られた型とその型代入を返す。
\\\\

\section{感想}
  レポート2までの課題に依存する課題が多かったためレポート2で実装していない部分についての拡張が行えずあまり必須課題以外をすることができなかった。これでしんどかった前期の計算機科学実験が終わると思うと感慨深いものがある。実際cpu実験もインタプリタ作成実験も非常に自分のためになりいい経験になったと思うが、自分の勉強時間が奪われたためもう少し分量が少なくてもいいかなとは思った。今実験ではまわりの友達に助けてもらうことが多く、自分を高めてくれる友達の存在には感謝せざる負えない。

\end{document}
